\documentclass{myarticle}

\title{Notes about Generative Adversarial Networks}
\author{赵浩翰}
\date{\today}

\begin{document}
    \maketitle
    \clearpage
    \tableofcontents
    \clearpage

    \section{Original GAN}

    \subsection{Introduction}
    Generative Adversarial Networks (GANs) \cite{goodfellow_generative_2014},生成对抗网络。

    生成模型,通过一个生成器 (Generator, G) 和一个鉴别器 (Discriminator, D) 的对抗性训练,G 用来估计真实数据的概率分布,D 用来估计样本来自真实数据的概率。G 的训练过程,即为最大化 D 的犯错概率。由此,G 和 D 之间形成了一个对抗性的博弈,G 努力学习真实数据分布,D 努力提升辨别真假数据分布的能力,形成一个 minimax 双人游戏。在 G 和 D 的任意函数空间中,存在唯一解 -- 纳什均衡,使得 G 重现真实数据分布,D 无法区分真假数据,即概率判断为 $\frac{1}{2}$。

    G 和 D 都是多层感知器 (Multilayer Perceptrons),G 的输入是一个随机噪声,输出是一个样本,D 的输入是一个样本,输出是一个概率值。二者可以通过后向传播 (Backpropagation) 进行训练,从而无需马尔科夫链 (Markov Chain) 以及近似推断 (Approximate Inference)。

    GAN 利用以下观察结果,研究生成过程中的反向传播导数:
    \begin{equation}
        \lim\limits_{\sigma\to0}\nabla_{\bm{x}}\mathbb{E}_{\epsilon\sim\mathcal{N}(0,\sigma^2\bm{I})}f(x+\epsilon)=\nabla_{\bm{x}}f(\bm{x})
    \end{equation}

    \subsection{Adversarial Nets}
    对抗性模型框架最直接的应用是其生成器 G 和鉴别器 D 都是多层感知器。为了学习生成器在数据 $\bm{x}$ 上的分布 $\bm{p_g}$,先验地定义一个输入的噪音变量 $p_{\bm{z}}(\bm{z})$,并将其在数据空间上的映射表示为 $G(\bm{z};\theta_g)$,其中 G 是一个以多层感知器表示的可微函数,参数为 $\theta_g$。同时,定义第二个多层感知器 $D(\bm{x};\theta_d)$,输出为一个标量,其中,$D{\bm{x}}$ 代表 $\bm{x}$ 来自真实数据而非 $p_g$ 的概率。

    训练 D 以最大化正确识别训练样本和来自生成器 G 的样本的概率,并同时训练 G 以最小化 $\log(1-D(G(\bm{z})))$。这个过程可以被看作是 D 和 G 的 minimax 二人游戏,其价值函数 $V(G,D)$ 为:
    \begin{equation}
        \min_G \max_D V(D,G)=\mathbb{E}_{\bm{x}\sim p_{data}(\bm{x})}[\log D(\bm{x})]+\mathbb{E}_{\bm{z}\sim p_{\bm{z}}(\bm{z})}[\log(1-D(G(\bm{z})))] \label{eq1: value function}
    \end{equation}

    当 G 和 D 被给予足够的容量,如无参数限制时,以上训练准则足以恢复真实数据分布。在实际中,需要使用迭代的数值方法执行以上 game 的训练过程。为避免有限数据集上的过拟合风险,需要在计算上禁止在训练的内循环中优化 D 直到结束。正确的方式应该是:迭代地训练 k 步 D,然后训练 1 步 G,如此重复。此时,只要 G 的改变足够缓慢,D 将会被维持在其最优解附近。训练过程如算法 \ref{alg1: Original GAN} 所示。

    \begin{algorithm}
        \caption{Original GAN}
        \label{alg1: Original GAN}
        Minibatch stochastic gradient descent training of generative adversarial nets.
        
        The number of steps to apply to the discriminator, k, is a hyperparameter.
        
        \begin{algorithmic}[1]
            \For {number of training iterations}
                \For {k steps}
                    \State Sample minibatch of $m$ noise samples $\{\bm{z}^{(1)},\bm{z}^{(2)},\cdots,\bm{z}^{(m)}\}$ from noise prior $p_{g}(\bm{z})$.
                    \State Sample minibatch of $m$ examples $\{\bm{x}^{(1)},\bm{x}^{(2)},\cdots,\bm{x}^{(m)}\}$ from data generating distribution $p_{data}(\bm{x})$.
                    \State Update the discriminator by ascending its stochastic gradient:
                    $$\nabla_{\theta_d}\frac{1}{m}\sum_{i=1}^m[\log D(\bm{x}^{(i)})+\log(1-D(G(\bm{z}^{(i)})))]$$
                \EndFor
                \State Sample minibatch of $m$ noise samples $\{\bm{z}^{(1)},\bm{z}^{(2)},\cdots,\bm{z}^{(m)}\}$ from noise prior $p_{g}(\bm{z})$.
                \State Update the generator by descending its stochastic gradient:
                $$\nabla_{\theta_g}\frac{1}{m}\sum_{i=1}^m\log(1-D(G(\bm{z}^{(i)})))$$
            \EndFor
        \end{algorithmic}
    \end{algorithm}

    在实践中,\ref{eq1: value function} 可能会导致 G 的梯度消失。在训练的早期,G 的能力较弱,D 可以轻松的识别真、假样本,导致 $\log (1-D(G(\bm{z})))\approx 0$,G 的梯度消失,训练速度缓慢。为了解决这个问题,可以在训练初期使用 $\log D(G(\bm{z}))$ 代替 $\log(1-D(G(\bm{z})))$,这个目标函数在G 和 D 相互作用时有相同的固定点,但在学习早期提供了更强的梯度。

    \subsection{Theoretical Results}
    
    如前所述,真实的数据 $\bm{x}$ 服从某个特定的分布 $p_{data}(\bm{x})$,而生成器 G 隐含地为其生成的样本 $G(\bm{z}),\ \bm{z}\sim p_{\bm{z}}$ 定义了一个概率分布 $p_g$。因此,在训练过程中,G 的目标便是学习一个分布 $p_g$,使得 $p_g=p_{data}$,即两个分布的“距离”越近越好,由此产生三个问题:
    \begin{enumerate}
        \item 如何度量两个分布的“距离”?
        \item $p_g(\bm{z})=p_{data}(\bm{x})$ 是否为生成器 G 的全局最优解?
        \item 上述训练算法是否可以使得 $p_g(\bm{z})$ 收敛于 $p_{data}(\bm{x})$?
    \end{enumerate}

    \subsubsection{KL \& JS Divergence}

    KL 散度 (Kullback-Leibler Divergence) 和 JS 散度 (Jensen-Shannon Divergence) 是用于度量两个分布之间的“距离”的方法。

    对于两个连续的概率分布 $p,q$,KL 散度定义为:
    \begin{equation}
        KL(p||q)=\int_{-\inf}^{\inf} p(\bm{x})\log\frac{p(\bm{x})}{q(\bm{x})}d\bm{x}
    \end{equation}

    KL 散度具有非负性,当两个分布完全相同,对于任意 $\bm{x}$,有 $p(\bm{x})=q(\bm{x})$,此时 $\log\frac{p(\bm{x})}{q(\bm{x})}=0$,KL 散度为 0。当两个分布不完全相同,根据吉布斯不等式 (Gibbs' Inequality) 可证明 KL 散度为正数。注意到 KL 散度不满足对称性,即 $KL(p||q)\neq KL(q||p)$。

    JS 散度解决了 KL 散度不对称的问题,定义为:
    \begin{equation}
        JS(p||q)=\frac{1}{2}KL(p||\frac{p+q}{2})+\frac{1}{2}KL(q||\frac{p+q}{2})
    \end{equation}

    JS 散度为两项 KL 散度之和,当 $p,q$ 两个分部完全相同,两项 KL 散度均为 0,JS 散度为 0。JS 散度同样满足非负性。JS 散度与 KL 散度的不同之处在于: (1) KL 散度无上界,而 JS 散度有上界 $\log2$;(2) JS 散度满足对称性,即 $JS(p||q)=JS(q||p)$。

    \subsubsection{Global Optimality of \texorpdfstring{$p_g=p_{data}$}{Lg}}

    考虑任意给定的生成器 G 下的最优鉴别器 D。
    \begin{proposition}[对于给定的生成器 G,最优鉴别器 D 为:]
        \begin{equation}
            D_{\bm{G}}^{*}(\bm{x})=\frac{p_{data}(\bm{x})}{p_{data}(\bm{x})+p_g(\bm{x})} \label{eq2: optimal discriminator}
        \end{equation}
    \end{proposition}

    \begin{proof}
        对于任意给定的 G,最优鉴别器 D 的目标是最大化价值函数 $V(D,G)$:
        \begin{equation}
            \begin{aligned}
                V(D,G)&=\int_{\bm{x}} p_{data}(\bm{x})\log(D(\bm{x}))d\bm{x}+\int_{\bm{z}} p_{\bm{z}}(\bm{z})\log(1-D(G(\bm{z})))d\bm{z} \\
                &=\int_{\bm{x}}[p_{data}(\bm{x})\log(D(\bm{x}))+p_g(\bm{x})\log(1-D(\bm{x}))]d\bm{x}
            \end{aligned}
        \end{equation}

        对于任意的 $(a,b)\in \mathbb{R}^2 \backslash \{0,0\}$,在 $[0,1]$ 区间上,函数 $y\rightarrow a\log(y)+b\log(1-y)$ 在点 $\frac{a}{a+b}$ 处取得最大值。同时,鉴别器无需在 $Supp(p_{data})\cup Supp(p_{g})$ 之外定义。由此得证。
        
        \QED
    \end{proof}

    由于 D 的训练目标可以视作最大化估计条件概率 $P(Y=y|\bm{x})$ 的对数似然,其中 $Y$ 为二值随机变量,表示样本来自真实数据 ($y=1$ when $\bm{x}\sim p_{data}$) 或生成器 G ($y=0$ when $\bm{x}\sim p_{g}$)。因此,\eqref{eq1: value function} 中的 minimax 游戏可以重新表述为:
    \begin{equation}
        \begin{split}
            C(G)&=\max_{D}V(G,D) \\
            &=\mathbb{E}_{\bm{x}\sim p_{data}}[\log D_{G}^{*}(\bm{x})]+\mathbb{E}_{\bm{z}\sim p_{\bm{z}}}[\log(1-D_{G}^{*}(G(\bm{z})))] \\
            &=\mathbb{E}_{\bm{x}\sim p_{data}}[\log D_{G}^{*}(\bm{x})]+\mathbb{E}_{\bm{x}\sim p_{g}}[\log(1-D_{G}^{*}(\bm{x}))] \\
            &=\mathbb{E}_{\bm{x}\sim p_{data}}[\log \frac{p_{data}(\bm{x})}{p_{data}(\bm{x})+p_g(\bm{x})}]+\mathbb{E}_{\bm{x}\sim p_{g}}[\log \frac{p_g(\bm{x})}{p_{data}(\bm{x})+p_g(\bm{x})}] \label{eq3: reformulated training criterion}
        \end{split}
    \end{equation}

    \begin{theorem}
        虚拟训练准则 $C(G)$ 的全局最小值当且仅当 $p_g=p_{data}$ 时取得。在该点处,$C(G)=-\log4$。 \label{thm1: global optimality of G}
    \end{theorem}

    \begin{proof}
        对于 $p_g=p_{data}$,$D_{G}^{*}(\bm{x})=\frac{1}{2}$ \eqref{eq2: optimal discriminator},由此,根据 \eqref{eq3: reformulated training criterion},有 $C(G)=\log\frac{1}{2}+\log\frac{1}{2}=-\log4$。

        对于任意 $p_g\neq p_{data}$,首先,将 $C(G)$ 的期望改写为积分形式:
        \begin{equation}
            \begin{split}
                C(G)&=\int_{\bm{x}} \left [p_{data}(\bm{x})\log \frac{p_{data}(\bm{x})}{p_{data}(\bm{x})+p_g(\bm{x})}+p_g(\bm{x})\log \frac{p_g(\bm{x})}{p_{data}(\bm{x})+p_g(\bm{x})}\right ]d\bm{x} \\
                &=\int_{\bm{x}} \bigg\{p_{data}(\bm{x})\left [-\log2+\log \frac{p_{data}(\bm{x})}{p_{data}(\bm{x})+p_g(\bm{x})}+\log2 \right ] \\ 
                &\qquad +p_g(\bm{x})\left [-\log2+\log \frac{p_g(\bm{x})}{p_{data}(\bm{x})+p_g(\bm{x})}+\log2 \right ]\bigg\} d\bm{x}
            \end{split}
        \end{equation}

        移项可得:
        \begin{equation}
            \begin{split}
                C(G)=&-\log2\int_{\bm{x}}\left [p_{data}(\bm{x})+p_g(\bm{x})\right ]d\bm{x} \\
                &+\int_{\bm{x}}p_{data}(\bm{x})\log\left [ \frac{p_{data}(\bm{x})}{(p_{data}(\bm{x})+p_g(\bm{x}))/2}\right ]d\bm{x} \\
                &+\int_{\bm{x}}p_g(\bm{x})\log\left [ \frac{p_{g}(\bm{x})}{(p_{data}(\bm{x})+p_g(\bm{x}))/2}\right ]d\bm{x} \\
                =&-2\log2+KL(p_{data}||\frac{p_{data}+p_g}{2})+KL(p_g||\frac{p_{data}+p_g}{2}) \\
                =&-\log4+2JS(p_{data}||p_g)
            \end{split}
        \end{equation}

        由于 JS 散度非负,当且仅当 $p_g=p_{data}$ 时,JS 散度取最小值 0,。此时,$C(G)$ 取得全局最小值 $-\log4$。因此,$p_g=p_{data}$ 是生成器 G 的全局最优解的充要条件。
        
        \QED
    \end{proof}

    \subsubsection{Convergence of Algorithm \ref{alg1: Original GAN}}

    \begin{proposition}
        当 G 和 D 有足够的容量,并且在 Algorithm \ref{alg1: Original GAN} 的每一步训练中,D 可以达到给定 G 下的最优状态,且 $p_g$ 以提升以下准则为目标进行更新时,$p_g$ 收敛于 $p_{data}$。
        \begin{equation}
            \mathbb{E}_{\bm{x}\sim p_{data}} \left [\log D_G^{*}(\bm{x})\right ] + \mathbb{E}_{\bm{x}\sim p_g} \left [\log (1-D_G^{*}(\bm{x}))\right ]
        \end{equation}
    \end{proposition}

    \begin{proof}
        当以上述准则进行训练时,$V(G,D)$ 可以视作 $p_g$ 的函数 $U(p_g,D)$。由于 $D$ 可以达到给定 $G$ 下的最优状态,则 $U(p_g,D)$ 是 $p_g$ 的凸函数。凸函数的上确界的子导数包括函数在最大值处的导数。即:如果 $f(x)=sup_{\alpha\in\mathcal{A}}f_{\alpha}(x)$ 且 $f_{\alpha}(x)$ 对任意 $\alpha$ 在 $x$ 上是凸函数,那么 $\partial f_{\beta}(x)\in\partial f\ when\ \beta=argsup_{\alpha\in\mathcal{A}}f_{\alpha}(x)$。这相当于在给定相应的 G 的最优的 D 下计算 $p_g$ 的梯度下降更新。$sup_D U(p_g,D)$ 对 $p_g$ 是凸函数,且由定理 \ref{thm1: global optimality of G} 可得其有唯一全局最优解,因此,当 $p_g$ 不断地、足够小幅度地更新时,$p_g$ 收敛于 $p_{data}$。
        
        \QED
    \end{proof}

    \rmk 在实践中,对抗网络通过函数 $G(z;\theta_g)$ 代表了一个 $p_g$ 的有限分布族,我们优化的是 $\theta_g$ 而不是 $p_g$ 本身,所以证明并不适用。

    \clearpage
    \bibliographystyle{unsrt}
    \bibliography{Reference}
\end{document}
